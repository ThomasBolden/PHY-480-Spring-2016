%!TEX program = xelatex
\documentclass[11pt]{article}

\usepackage{amsmath,  latexsym, amssymb, url, amssymb, pgfplots, amsthm, mathtools, setspace, nopageno, commath, tikz, verbatim, array, enumitem, bbding, bigints, fontspec, xunicode, xltxtra, geometry, graphicx,listings,lipsum,tabto,tcolorbox,sectsty,booktabs,siunitx,caption, color}
\usepackage[version=4]{mhchem}
\usepackage[utf8]{inputenc}

\NumTabs{6}

\defaultfontfeatures{Mapping=tex-text} 

\setmainfont{Charter} 
%\setsansfont{Deja Vu Sans}
%\setmonofont{Deja Vu Mono}

\definecolor{gry}{rgb}{.95,.95,.95}

\lstset{language=C++,
                basicstyle=\ttfamily,
                keywordstyle=\color{magenta}\ttfamily,
                stringstyle=\color{green}\ttfamily,
                commentstyle=\color{gray}\ttfamily,
                morecomment=[l][\color{orange}]{\#}}
\lstset{alsolanguage=[90]Fortran}
\lstset{basicstyle=\small}
\lstset{backgroundcolor=\color{gry}}
\lstset{frame=single}
\lstset{
    numbers=left,
    title=\lstname,
    tabsize=4,
}



\geometry{margin=1in}
\geometry{letterpaper} % or letterpaper (US) or a5paper or....
\usepackage[parfill]{parskip} % Activated to begin paragraphs with an empty line rather than an indent

\title{PHY 480 - Computational Physics \\ Project 1: Linear Algebra Methods}
\author{Thomas Bolden}
\date{February 12, 2016}

\usepackage{hyperref}
\hypersetup{
    colorlinks=true,
    linkcolor=darkgray,
    filecolor=magenta,      
    urlcolor= gray,
}

\setcounter{secnumdepth}{0} % deactivate for section numbers
%\sectionfont{\fontsize{12}{15}\selectfont} % Activate for same font size sections
\newenvironment{amatrix}[1]{%
  \left[\begin{array}{@{}*{#1}{c}|c@{}}
}{%
  \end{array}\right]
}

\begin{document}

%\tableofcontents
\maketitle

\centerline{Github Repository at \href{https://github.com/ThomasBolden/PHY-480-Spring-2016}{https://github.com/ThomasBolden/PHY-480-Spring-2016}}

\begin{abstract}

    herrings

\end{abstract}

\vspace{\fill}
\tableofcontents

\pagebreak

\section{Introduction}

    An important part of physics is being able to efficiently solve systems of linear equations. . . Given a differential equation of the form 
    \begin{equation} - \dfrac{\dif\,^2}{\dif x^2} u(x) = f(x) \end{equation}
    where $f(x)$ is continuous on the domain $x \in (0,1)$. We also assume the boundary conditions $u(0)=u(1)=0$. 

\section{Methods}

    .

\section{Results}

    .

\section{Conclusions}

    .

\section{Code}

    \lstinputlisting[language=C++]{../Code/Project1.cpp}

\begin{thebibliography}{1}

    \bibitem{morten}

\end{thebibliography}














\end{document}