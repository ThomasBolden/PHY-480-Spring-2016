%!TEX program = xelatex
\documentclass[11pt]{article}

\usepackage[dvipsnames]{xcolor} % needed to declare before tikz
\usepackage{amsmath,  latexsym, amssymb, url, amssymb, pgfplots, amsthm, mathtools, setspace, commath, tikz, verbatim, array, enumitem, bbding, bigints, fontspec, xunicode, xltxtra, geometry, graphicx,listings,lipsum,tabto,tcolorbox,sectsty,booktabs,siunitx,caption,}
\usepackage[version=4]{mhchem}
\usepackage[utf8]{inputenc}


\NumTabs{6}

\defaultfontfeatures{Mapping=tex-text} 

\setmainfont{Charter} 

\newfontfamily\Codefont{Courier}  %{GillSans} %{AmericanTypewriter}

\definecolor{gry}{rgb}{.95,.95,.95}

\lstset{language=C++,
                basicstyle=\Codefont,
                identifierstyle=\Codefont,
                directivestyle=\color{BurntOrange}\Codefont,
                keywordstyle=\color{RubineRed}\Codefont,
                stringstyle=\color{Green}\Codefont,
                commentstyle=\color{gray}\Codefont,
                emphstyle=\Codefont,
                showspaces=false,
                morecomment=[l][\color{orange}]{\#}}
%\lstset{alsolanguage=[90]Fortran}
\lstset{basicstyle=\small}
\lstset{backgroundcolor=\color{gry}}
\lstset{frame=single}
\lstset{
    numbers=left,
    numberstyle=\Codefont,
    title=\lstname,
    tabsize=4,
}



\geometry{margin=1in}
\geometry{letterpaper} % or letterpaper (US) or a5paper or....
\usepackage[parfill]{parskip} % Activated to begin paragraphs with an empty line rather than an indent

\title{PHY 480 - Computational Physics \\ Project 1: Linear Algebra Methods}
\author{Thomas Bolden}
\date{February 12, 2016}

\usepackage{hyperref}
\hypersetup{
    colorlinks=true,
    linkcolor=darkgray,
    filecolor=magenta,      
    urlcolor=Emerald,
}

\setcounter{secnumdepth}{0} % deactivate for section numbers
%\sectionfont{\fontsize{12}{15}\selectfont} % Activate for same font size sections
\subsectionfont{\fontsize{15}{15}\selectfont}
\newenvironment{amatrix}[1]{%
  \left[\begin{array}{@{}*{#1}{c}|c@{}}
}{%
  \end{array}\right]
}

\begin{document}

%\tableofcontents
\maketitle

\thispagestyle{empty}

\centerline{Github Repository at \href{https://github.com/ThomasBolden/PHY-480-Spring-2016}{https://github.com/ThomasBolden/PHY-480-Spring-2016}}

\begin{abstract}

    %In this project, a solution was found to the one-dimensional Poissson equation with Dirichlet boundary conditions. This was done by rewriting the Poisson equation as a set of linear equations and implimenting a tridiagonal matrix solver. The error in the results ... 
    \lipsum[1-1]

\end{abstract}

\vspace{\fill}
\tableofcontents

 \vspace{2cm}

\pagebreak

\setcounter{page}{1}

\subsection{Introduction}

    An important part of physics is being able to efficiently solve systems of linear equations. . . 

\subsection{Methods}

    Given a differential equation of the form 
    \begin{equation} - \dfrac{\dif\,^2}{\dif x^2} u(x) = f(x) \end{equation}
    where $f(x)$ is continuous on the domain $x \in (0,1)$. We also assume the boundary conditions $u(0)=u(1)=0$. The second derivative can be approximated as 
    \begin{equation} u'' = \dfrac{u_{i+1}+u_{i-1}-2u_i}{u^2} \end{equation}

    \[ \arraycolsep 1.0ex \bold{A} = \left( \begin{array}{c c c c c c c} 
    \vspace{.1cm} 2 & \llap{-}1 & 0 & \cdots & \cdots & \cdots & 0 \\
    \llap{-}1 & 2 & \llap{-}1 & 0 & \cdots & \cdots & 0 \\
    0 & \llap{-}1 & 2 & \llap{-}1 & \ddots &  & 0 \\
    \vdots & 0 & \llap{-}1 & 2 & \ddots & \ddots & \vdots \\
    \vdots & \vdots & \ddots &\ddots&\ddots&\ddots&0 \\
    \vdots &\vdots&&\ddots&\ddots&\ddots& \llap{-}1 \vspace{.1cm} \\ 
    0 &0&\cdots&\cdots&0&\llap{-}1& 2 
    \end{array} \right) \;\; , \;\; \bold{v} = \left( \begin{array}{c} v_0 \\ v_1 \\ \vdots \\ v_{n-1} \\ v_n \end{array} \right) \]

\subsection{Results}

    .

\subsection{Conclusions}

    .

\subsection{Code}

    \lstinputlisting[language=C++]{../Code/Project1.cpp}

\begin{thebibliography}{1}

    \bibitem{morten} M. Hjorth-Jensen, {\em Computational Physics}, University of Oslo (2013).

    \bibitem{mclean} W. McLean, {\em Poisson Solvers}, Northwestern University (2004).

\end{thebibliography}














\end{document}