%!TEX program = xelatex
\documentclass[11pt]{article}

\usepackage[dvipsnames]{xcolor}
\usepackage{amsmath,  latexsym, amssymb, url, amssymb, pgfplots, amsthm, mathtools, setspace, nopageno, commath, tikz, verbatim, array, enumitem, bbding, bigints, fontspec, xunicode, xltxtra, geometry, graphicx, listings, lipsum, tabto, tcolorbox, sectsty, booktabs, siunitx, caption, algorithm, algpseudocode}
\usepackage[version=4]{mhchem}
\usepackage[utf8]{inputenc}
\usepackage[parfill]{parskip}
\usepackage{hyperref}
\hypersetup{
    colorlinks=true,
    linkcolor=darkgray,
    filecolor=magenta,      
    urlcolor= JungleGreen,
}

\geometry{margin=1in}
\geometry{letterpaper} % or letterpaper (US) or a5paper or....

\NumTabs{6}

\defaultfontfeatures{Mapping=tex-text} 

\setmainfont{Charter} 
\setsansfont{Charter}
\setmonofont{Charter}

\lstset{language=C++,
                moredelim=[is][\color{Green}\Codefont]{<}{>},
                basicstyle=\small\Codefont,
                identifierstyle=\Codefont,
                directivestyle=\color{BurntOrange}\Codefont,
                keywordstyle=\color{RubineRed}\Codefont,
                stringstyle=\color{Green}\Codefont,
                commentstyle=\color{gray}\Codefont,
                emphstyle=\color{Violet}\Codefont,
                emph={1,2,3,4,5,6,7,8,9,0},
                showspaces=false,
                %keywordstyle=[2]\color{Violet},
                keywords=[2]{*,1,2,3,4,5,6,7,8,9,0},
                keywordstyle=[2]\color{blue},
                showstringspaces=false,
                morecomment=[l][\color{orange}]{\#}}
\lstset{literate=%
    *{0}{{{\color{NavyBlue}0}}}1
    {1}{{{\color{NavyBlue}1}}}1
    {2}{{{\color{NavyBlue}2}}}1
    {3}{{{\color{NavyBlue}3}}}1
    {4}{{{\color{NavyBlue}4}}}1
    {5}{{{\color{NavyBlue}5}}}1
    {6}{{{\color{NavyBlue}6}}}1
    {7}{{{\color{NavyBlue}7}}}1
    {8}{{{\color{NavyBlue}8}}}1
    {9}{{{\color{NavyBlue}9}}}1
    {.0}{{{\color{NavyBlue}.0}}}2
    {.1}{{{\color{NavyBlue}.1}}}2
    {.2}{{{\color{NavyBlue}.2}}}2
    {.3}{{{\color{NavyBlue}.3}}}2
    {.4}{{{\color{NavyBlue}.4}}}2
    {.5}{{{\color{NavyBlue}.5}}}2
    {.6}{{{\color{NavyBlue}.6}}}2
    {.7}{{{\color{NavyBlue}.7}}}2
    {.8}{{{\color{NavyBlue}.8}}}2
    {.9}{{{\color{NavyBlue}.9}}}2
}
\lstset{alsolanguage=[90]Fortran}
\lstset{alsolanguage=python}
\lstset{backgroundcolor=\color{gry}}
\lstset{frame=single}
\lstset{
    numbers=left,
    numberstyle=\Codefont,
    title=\lstname,
    tabsize=4,
}

\title{PHY 480 - Computational Physics \\ Project 2: Schr\"{o}dinger's Equation for 2 electrons in a 3D Well}
\author{Thomas Bolden}
\date{March 4, 2016}


\setcounter{secnumdepth}{0} 
%\sectionfont{\fontsize{12}{15}\selectfont} 
\newenvironment{amatrix}[1]{
  \left[\begin{array}{@{}*{#1}{c}|c@{}}
}{
  \end{array}\right]
}

\begin{document}

\maketitle

\centerline{Github Repository at \href{https://github.com/ThomasBolden/PHY-480-Spring-2016}{https://github.com/ThomasBolden/PHY-480-Spring-2016}}

\begin{abstract}
\lipsum[1-1]
\end{abstract}

\vfill

\tableofcontents

\vspace{3cm}

\pagebreak

\subsection{Introduction}

    In physics, there is a known solution to Schr\"{o}dinger's equation for a single electron in a spherically symmetric three-dimensional well. However, this equation becomes more complicated when another elctron is added. In addition to the energy term, there are Coulombic interactions between the electrons that must be accounted for. 

    For one electron, the radial part of the Schr\"{o}dinger equation reads
    \[ -\dfrac{\hbar^2}{2m} \left( \dfrac{1}{r^2} \dfrac{d}{dr}r^2\dfrac{d}{dr} - \dfrac{\ell (\ell -1)}{r^2} \right) R(r) + V(r) R(r) = E R(r) \;\; , \;\; V(r) = \dfrac{m\omega^2 r^2}{2}. \]

    % Structure of the report

\subsection{Methods}

    .

\subsection{Results}

    .

\subsection{Conclusions}

    .

\subsection{Code}

    .

\begin{thebibliography}{1}

\bibitem{morten} 
    M. Hjorth-Jensen, {\em Computational Physics}, University of Oslo (2013).

\end{thebibliography}



















\end{document}